% Disable single lines at the start of a paragraph (Schusterjungen)
\clubpenalty=10000

% Disable single lines at the end of a paragraph (Hurenkinder)
\widowpenalty=10000
\displaywidowpenalty=10000

% Fixed size table columns
\newcolumntype{L}[1]{>{\raggedright\arraybackslash}p{#1}}
\newcolumntype{C}[1]{>{\centering\arraybackslash}p{#1}}
\newcolumntype{R}[1]{>{\raggedleft\arraybackslash}p{#1}}

\renewcommand{\arraystretch}{1.2}                       % Distance between lines in tables

\captionsetup{justification=raggedright}                % Align captions to the left

\setcounter{secnumdepth}{3}                             % Only allow nesting 3 layers (down to subsubsections)

% Unit-related setup for siunitx
\sisetup{
  locale=DE,                   % Use German notation (comma instead of point delimiter for floats)
  per-mode=fraction,           % Switch display to use \frac instead of x^{-1}
  fraction-function=\tfrac,    % Use amsmath's tfrac macro for unit fractions
}

% -------------------------------------------------------------------
%                     Usability & visual changes
% -------------------------------------------------------------------

% Page margins
\geometry{
  left=2.5cm,
  right=2.5cm,
  top=2.5cm,
  bottom=3cm
}

% Create a better looking header and footer
\pagestyle{fancy}
\fancyhf{}
\lhead{\nouppercase{\leftmark}}
\rfoot{\thepage}

% Automatically generate a box around figure environments
\floatstyle{boxed}
\restylefloat{figure}

% Set toc sections to be clickable
\hypersetup{
  colorlinks,
  citecolor=black,
  filecolor=black,
  linkcolor=black
}

\frenchspacing                                          % Insert one space after a sentence, not 2

\renewcommand{\UrlFont}{\color{blue}\rmfamily\itshape}  % URLs should be displayed in blue
\renewcommand{\dateseparator}{.}                        % Dates are written like 01.01.1970, not 01-01-1970

\addto{\captionsngerman}{
  \renewcommand*{\figurename}{Abb.}                     % Figures should be displayed as "Abb. x"
  \renewcommand*{\tablename}{Tab.}                      % Tables  should be displayed as "Tab. x"
  \renewcommand*{\lstlistingname}{Code}                 % Code    should be displayed as "Code x"
  \renewcommand*{\nomname}{Symbolverzeichnis}           % Nomenclature in German is "Symbolverzeichnis"
}

% -------------------------------------------------------------------
%                        Code listing setup
% -------------------------------------------------------------------

\lstset{
  basicstyle=\small\ttfamily\color{black},              % Font size used for the code
  commentstyle=\ttfamily\color{gray},                   % Comment style
  keywordstyle=\ttfamily\color{blue},                   % Keyword style
  stringstyle=\color{ForestGreen!30!LimeGreen},         % String literal style
  frame=single,                                         % Add a frame around the code
  showstringspaces=false,                               % Don't underline spaces within strings only
  captionpos=b,                                         % Set caption-position to bottom
  backgroundcolor=\color{white},                        % Background color
}

% To style lstlistlisting like the lof, you first have to register it
% to tocloft, as mentioned in https://tex.stackexchange.com/a/27648/27635
\makeatletter
\begingroup\let\newcounter\@gobble\let\setcounter\@gobbletwo
  \globaldefs\@ne \let\c@loldepth\@ne
  \newlistof{listings}{lol}{\lstlistlistingname}
\endgroup
\let\l@lstlisting\l@listings
\makeatother

% -------------------------------------------------------------------
%                         Redefining geometry
% -------------------------------------------------------------------

% Figures
\renewcommand{\cftfigpresnum}{\figurename~}
\renewcommand{\cftfigaftersnum}{:}
\setlength{\cftfignumwidth}{2cm}
\setlength{\cftfigindent}{0cm}

% Tables
\renewcommand{\cfttabpresnum}{\tablename~}
\renewcommand{\cfttabaftersnum}{:}
\setlength{\cfttabnumwidth}{2cm}
\setlength{\cfttabindent}{0cm}

% Listings
\renewcommand*{\cftlistingspresnum}{\lstlistingname~}
\renewcommand*{\cftlistingsaftersnum}{:}
\settowidth{\cftlistingsnumwidth}{\cftlistingspresnum}
\addtolength{\cftlistingsnumwidth}{1cm}
\setlength{\cftlistingsindent}{0cm}

\setlength{\parindent}{0cm}                             % Don't indent start of paragraph
\setlength{\parskip}{6pt}                               % Lines are seperated by 6pt

\setlength{\headheight}{1.25cm}
\setlength{\footskip}{1cm}
\setlength{\headsep}{1cm}

% -------------------------------------------------------------------
%                    Custom counters & commands
% -------------------------------------------------------------------

\newcounter{countacronym}
\DeclareTotalCounter{countacronym}
\newcommand*{\acr}[1]{\acrshort{#1}\stepcounter{countacronym}}
\newcommand*{\Acr}[1]{\acrlong{#1}\stepcounter{countacronym}}

\newcounter{countnomen}
\DeclareTotalCounter{countnomen}
\newcommand*{\nomen}[2]{\nomenclature{#1}{#2}\stepcounter{countnomen}}
\newcommand{\nomunit}[1]{\renewcommand{\nomentryend}{\hspace*{\fill}#1}}
\newcommand{\nomsi}[1]{\nomunit{[\si{#1}]}}

% Count number of references to the glossary
\newcounter{countglossary}
\DeclareTotalCounter{countglossary}
\pretocmd{\Gls}{\stepcounter{countglossary}}{}{}
\pretocmd{\gls}{\stepcounter{countglossary}}{}{}
\pretocmd{\Glspl}{\stepcounter{countglossary}}{}{}
\pretocmd{\glspl}{\stepcounter{countglossary}}{}{}

% TODO annotations
\newcounter{counttodo}
\DeclareTotalCounter{counttodo}
\newcommand{\note}[2][]{
  \todo[color=green!25,bordercolor=green,tickmarkheight=3pt,#1]{#2}
  \stepcounter{counttodo}
}
\newcommand{\unsure}[2][]{
  \todo[color=Plum!25,bordercolor=Plum,tickmarkheight=3pt,#1]{#2}
  \stepcounter{counttodo}
}
\newcommand{\change}[2][]{
  \todo[color=blue!25,bordercolor=blue,tickmarkheight=3pt,#1]{#2}
  \stepcounter{counttodo}
}

\newcommand{\imgref}[1]{\hyperref[#1]{Abbildung~\getrefnumber{#1}}}
\newcommand{\tabref}[1]{\hyperref[#1]{Tabelle~\getrefnumber{#1}}}
\newcommand{\coderef}[1]{\hyperref[#1]{Code~\getrefnumber{#1}}}
\newcommand{\mathref}[1]{\hyperref[#1]{Gleichung~\getrefnumber{#1}}}
\newcommand{\secref}[1]{\hyperref[#1]{Abschnitt~\getrefnumber{#1}}}
