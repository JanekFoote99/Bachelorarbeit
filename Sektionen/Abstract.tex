\selectlanguage{english}
\begin{abstract}
3D models play a central role in the video game and animation film industry. 
They offer artists the opportunity to model characters and environments that simulate a representation of reality. \newline
In computer graphics, triangle meshes are a common data representation of 3D-models. 
However, the models are becoming increasingly complex, which increases the memory requirements and extends the time needed for visualisation. To meet these requirements, new methods for data compression must be developed. The aim of this work is to develop a method for loading a compressed triangle mesh into the GPU and decoding it there. In particular, the compression ratio, decompression rate and visual quality are to be quantitatively analysed and evaluated. If possible, the results of this work will be compared with the results of common compression methods.
\end{abstract}

\selectlanguage{ngerman}
\begin{abstract}
In der Videospiel- und Animationsfilmindustrie spielen 3D-Modelle eine zentrale Rolle. 
Sie bieten einem Künstler die Möglichkeit Charakteren und Umgebungen zu modellieren, die eine Darstellung der Realität simulieren soll. \newline
In der Computergrafik sind Dreiecksnetzes eine gängige Datenrepräsentation von 3D-Modellen. 
Die Modelle werden jedoch immer komplexer, was den Speicherbedarf erhöht und die zur Darstellung benötigte Zeit verlängert. Um diesen Anforderungen gerecht zu werden, müssen neue Methoden zur Datenkompression entwickelt werden. Ziel dieser Arbeit ist es, ein Verfahren zu entwickeln, um ein komprimiertes Dreiecksnetz in die GPU zu laden und dort zu dekodieren. Insbesondere sollen das Kompressionsverhältnis, Dekompressionsrate und visuelle Qualität quantitativ untersucht und auswertet werden. Die Ergebnisse dieser Arbeit werden, wenn möglich, mit den Resultaten von gängigen Kompressionsmethoden verglichen.
\end{abstract}