\section{Fazit}
\label{sec:fazit}
Die Ergebnisse sind 

Die Bandbreite scheint höher auszufallen, je größer die komprimierte Größe des Dreiecksnetzes ist.

Wie in den Ergebnissen zu sehen bietet die Quantisierung dazu noch eine nicht zu verachtende Reduktion der Speichergröße.
Die Kombination mit der Kompression mit Brotli-G bietet dazu noch bessere Ergebnisse der komprimierten Größe, aufgrund von dem nicht kodieren müssen der niederwertigen Bits, die nur rauschen verursachen.

\subsection{Bewertung der Ergebnisse}
\label{subsec:bewertung}
Tiefpassfilter

Eine Möglichkeit für die geringe Bandbreite scheint auf dem ersten Blick die Größe der komprimierten Daten zu sein.
Auffällig ist nämlich die geringe Bandbreite bei kleinen Dreiecksnetzen wie der Fandisk.
Mit einer komprimierten Größe von 0,160 MB benötigt diese am wenigsten Speicherplatz, und erreicht eine Bandbreite von 0,054 GiB/s.
Weitere Beispiele sind das Bunny mit einer Größe von 1,067 MB und einer Bandbreite von 0,503 GiB/s, der Dinosaur mit 1,713 MB und einer Bandbreite von 0,825 GiB/s und der Rockerarm mit 1,139 MB und einer Bandbreite von 0,399 GiB/s.
Es scheint dabei zudem eine untere Grenze der Dekompressionszeit zu geben.
Diese liegt bei den sehr kleinen Dreiecksnetzen wie den eben aufgeführten bei 2 ms - 3 ms.
Bei Betrachtung der mittelgroßen Dreiecksnetze des Datensatzes scheint diese nicht einheitlich anzusteigen.
Die Dekompressionszeit der Hand, des Armadillos und des Angels befindet liegt auch bei 3 ms.
Die Größe der komprimierten Dreiecksnetze liegt hierbei jedoch bei 8,195 MB, 4,486 MB und 7,239 MB.
Die Zeitrechnung des Brotli-G Dekodierers misst jedoch nicht nur die Dauer des Compute Shaders.
Der Umstand einer unteren Schranke bei 2 ms kann ebenso am Overhead der API Aufrufe von Directx12 kommen.
Die Zeitrechnung beginnt demnach bei dem Aufruf der decodeGPU Methode, und endet, nachdem der Destruktor das Objekt zerstört, das den Compute Shader mit Daten gefüttert und ausgeführt hat. \newpage

Der Großteil der Daten eines Dreiecksnetzes besteht aus seiner Geometrie \cite{Jakob2017}.
Bei Betrachtung der Komponenten ist zwar erkennbar, das es wesentlich mehr Dreiecke, und somit auch mehr Indizes, als Vertices gibt.
Wenn als Topologie eine \textit{Triangle List} verwendet wird, gelangt man auf ein Verhältnis von 
\begin{equation*}
V:T = 1:2
\end{equation*}
Da ein Dreieck aus 3 Indizes besteht, enthält ein Dreiecksnetz von dieser Topologie in etwa 6 Indizes pro Vertex (Euler's Formel) \cite{Engstad2011}.
Angenommen, ein Vertex bestehe lediglich aus Positionen und Normalen, wie es bei dem verwendeten Datensatz dieser Arbeit der Fall war.
So enthält ein Vertex ohne Kompression der Daten
\begin{equation*}
3 * 4 \ \text{Byte} = 12 \ \text{Byte}
\end{equation*}
Gleitkommazahlen für die Position im 3-dimensionalen Raum, und nochmal 12 Byte für die Normalen.
So besteht ein Vertex aus 24 Byte, bzw. 192 Bit Daten.
Ein unkomprimierter Index wird in einem Integer ohne Vorzeichen gespeichert, der 4 Byte, bzw. 32 Bit beansprucht \cite{Microsoft2021a}. \newline
Beachtet man das Verhältnis der Euler Formel bei Dreiecksnetzen, enthalten 6 Indizes mit 192 Bit die gleiche Datenmenge wie ein Vertex mit Position und Normale.
Zu beachten ist jedoch, dass neben der Position auch Texturkoordinaten, Farben, Tangenten und Bitangenten in den Vertex Attributen vorhanden sein können. \newline

Jedenfalls ist erkennbar, dass die Vertices einen maßgeblichen Einfluss auf die Datengröße nehmen.
Bei jedem einzelnen Dreiecksnetz des Datensatzes ist eine Reduktion der Daten von 28,4\% zu sehen.
Die Quantisierung mindert jedoch nicht nur die originalen Eingabedaten.
Vergleicht man die von Brotli-G komprimierten Originaldaten, und die von Brotli-G komprimierten Daten seines quantisierten Gegenüber, so ist eine Reduktion der komprimierten Daten von durchschnittlich 39,71\% möglich.
Die meisten Dreiecksnetze sind in einem Bereich von 38\% - 42\%.
Die Ausreißer liegen jedoch nicht weit entfernt von diesem Bereich, mit dem Armadillo der mit 32,6\% am schlechtesten Abgeschnitten hat, während der Welsh Dragon mit 45,6\% die größte Reduktion der Originaldaten hat.
Diese Diskrepanz der Prozente ist gut erkennbar, wenn die Kompressionsverhältnisse von quantisierten und nicht quantisierten Dreiecksnetzen betrachtet wird.
Hier wird schnell klar, das der Armadillo mit 6,42\% den geringsten Zuwachs der Kompressionsverhältnisse erlangt hat, während der Welsh Dragon mit 32,5\% den besten Zuwachs der Kompressionsverhältnisse verzeichnet.
Diese komprimierten Daten können direkt aus einem Speichermedium in den GPU RAM hochgeladen werden, wo sie vom Brotli-G Compute Shader dekomprimiert, und vom Mesh Shader gerendert werden.
Das hängt mit dem Rauschen der Daten zusammen.

\subsection{Ausblick}
\label{subsec:ausblick}
Kurz vor Abgabe gab es eine neue Brotli-G Version 

Zur zusätzlichen Datenreduktion bietet sich noch das packen der Indizes bei Meshlets

Wie zu sehen ist ergeben sich durch die Komprimierung mittels Brotli-G gute Kompressionsverhältnisse.
Die Auswertung der Ergebnisse ergibt zudem, das weitere Verfahren zur Datenkompression nicht nur zur direkten Datenreduktion führen kann, sondern zusätzlich die Komprimierung mit Brotli-G verbessern kann.
In dieser Arbeit wurde der Einsatz von der Quantisierung verwendet.
Andere Verfahren wie die prädiktive Kodierung 

Nach einem Blogpost auf GPUOpen von AMD, wurde die neue Version 1.1 von Brotli-G für den Compressonator Version 4.5 verwendet.
Die Tests die in dem Blogpost behandelt werden, richten sich zwar an Texturen, die Ergebnisse von der neuen Version mit Brotli-G 1.1 sind dennoch sehenswert.
Im Vergleich zur alten Version reduziert der Compressonator mit der neuen Brotli-G Version die komprimierten Texturen im Durchschnitt um weitere 10\% - 15\%, in besonders guten Fällen sogar um 20\% \cite{Levesque2024}.
Die Brotli-G Version 1.1 wurde leider erst gegen Ende dieser Arbeit veröffentlicht, wodurch die vorliegenden Ergebnisse auf der Brotli-G Version 1.0 bestehen.
Die besseren Ergebnisse des Compressonators lassen auf zukünftig noch bessere Resultate, und vor allem auf eine effektivere Ausnutzung der Bandbreite des PCI Express Busses hoffen.
Die Ergebnisse dieser Arbeit in der Hinsicht der Dekompressionszeit und Bandbreitenausnutzung sind selbst bei den größeren Dreiecksnetzen weit von gewünschten Werten entfernt.
Für die Versuche wurde die Grafikkarte \textit{Radeon RX 6650 XT} von AMD verwendet.
Diese kann unter optimalen Bedingungen eine maximale Bandbreite von 238 GiB/s erreichen.
Diese soll laut Datenbankeintrag eine maximale Bandbreite von 280 GB/s, also zum Vergleich umgerechnet ungefährt 260 GiB/s erreichen.

Zukünftig sollte die neue Brotli-G Version verwendet werden, damit bessere Ergebnisse zu erzielen.

