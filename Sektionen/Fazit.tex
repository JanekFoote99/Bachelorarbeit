\section{Fazit}
\label{sec:fazit}
Für die Versuche wurde eine \textit{RX 6650 XT} von AMD verwendet.
Diese kann unter optimalen Bedingungen eine maximale Bandbreite von 238 GiB/s erreichen.
Die Ergebnisse sind 

Die Bandbreite scheint höher auszufallen, je größer die komprimierte Größe des Dreiecksnetzes ist.

Wie in den Ergebnissen zu sehen bietet die Quantisierung dazu noch eine nicht zu verachtende Reduktion der Speichergröße.
Die Kombination mit der Kompression mit Brotli-G bietet dazu noch bessere Ergebnisse der komprimierten Größe, aufgrund von dem nicht kodieren müssen der niederwertigen Bits, die nur rauschen verursachen.

\subsection{Bewertung der Ergebnisse}
\label{subsec:bewertung}

Tiefpassfilter

\subsection{Ausblick}
\label{subsec:ausblick}
Kurz vor Abgabe gab es eine neue Brotli-G Version 

Zur zusätzlichen Datenreduktion bietet sich noch das packen der Indizes bei Meshlets

Wie zu sehen ist ergeben sich durch die Komprimierung mittels Brotli-G gute Kompressionsverhältnisse.
Die Auswertung der Ergebnisse ergibt zudem, das weitere Verfahren zur Datenkompression nicht nur zur direkten Datenreduktion führen kann, sondern zusätzlich die Komprimierung mit Brotli-G verbessern kann.
In dieser Arbeit wurde der Einsatz von der Quantisierung verwendet.
Andere Verfahren wie die prädiktive Kodierung 

Nach einem Blogpost auf GPUOpen von AMD, wurde die neue Version 1.1 von Brotli-G für den Compressonator Version 4.5 verwendet.
Die Tests die in dem Blogpost behandelt werden, richten sich zwar an Texturen, die Ergebnisse von der neuen Version mit Brotli-G 1.1 sind dennoch sehenswert.
Im Vergleich zur alten Version reduziert der Compressonator mit der neuen Brotli-G Version die komprimierten Texturen im Durchschnitt um weitere 10\% - 15\%, in besonders guten Fällen sogar um 20\%.
Die Brotli-G Version 1.1 wurde leider erst gegen Ende dieser Arbeit veröffentlicht, wodurch alle Ergebnisse dieser Arbeit auf der Version 1.0 von Brotli-G bestehen.
Zukünftig sollte die neue Brotli-G Version verwendet werden, um künftig bessere Ergebnisse zu erzielen.

